\documentclass[12pt,a4paper]{article}
\usepackage[margin=1in]{geometry}
\usepackage{booktabs}
\usepackage{setspace}
\usepackage{hyperref}


\onehalfspacing
\setcounter{secnumdepth}{4}

%=== COVER PAGE ===
\begin{document}
\begin{titlepage}
    \centering
    {\Large\bfseries CM10025 Programming 2: Personal Informatics System Report\par}
    \vspace{1cm}
    Group Name: Agile Analysts \\
    Group Number: 18 \\
    Date: March 24, 2025

    \vspace{2cm}
    \begin{tabular}{llll}
        \toprule
        Group Member & Username & Degree & Course \\
        \midrule
        David Cai & yc2800 & MComp Computer Science and Mathematics & Year 1 \\
        Mandeep Thakur & mt2434 &  MComp Computer Science and Mathematics & Year 1\\
        Jack Bancroft & jgb64 & BSc Computer Science & Year 1\\
        James Beagrie & jb4106 & BSc Computer Science and Mathematics & Year 1\\
        Satima Dosso & sd2745 & BSc Computer Science and Mathematics & Year 1 \\
    \end{tabular}
    \thispagestyle{empty}
\end{titlepage}

%=== TITLE + ABSTRACT (1 page) ===
\begin{center}
    {\Large\bfseries Personal Informatics System Report}\\[1ex]
    Author(s): David Cai, Mandeep Thakur, Jack Bancroft, James Beagrie, Satima Dosso
\end{center}
\begin{abstract}
    200 words summarising problem, solution, 3 sprint outcomes, next steps
    
\end{abstract}
\newpage

%=== TABLE OF CONTENTS ===
\tableofcontents
\newpage



%=== MAIN BODY (max 20 pages) ===

\section{Introduction (2 pages)}
Author: Mandeep Thakur
\subsection{Background Of The Problem}
University students often struggle with effective time management, which limits their academic performance and leads to procrastination. There are several studies that show a direct correlation between time management and academic achievement of university students [1]. First-year students, in particular, face the greatest amount of challenges [2]; getting familiar with a new environment and balancing both social and academic responsibilities along with the rise of digital distractions make it increasingly difficult for students to allocate their time efficiently. This situation has created the need for a Personal Informatics (PI) system to offer insights into time usage habits and track daily routines. Personal Informatics (PI) is a field focused on assisting people by gathering, analysing and reflecting on personal data in order to better understand their habits and behaviours [5]. These systems have emerged and convert data into actionable insights, enabling individuals to improve their behaviour by making their routines visible and quantifiable [3]. There is strong evidence that supports that PI systems affect user behaviour. According to research [4], 38 \% of studies stated that PI systems led to noticeable changes in behaviour, which demonstrates how these systems can potentially improve users' expectations for their own outcomes while raising awareness of their own habits.

\subsection{Effectiveness of PI software systems}

User involvement, data accuracy, and the capacity to convey information in an engaging and meaningful manner are some aspects that affect how effective these systems are. In order for a PI system to be effective, there must be several features the system must follow. It must provide meaningful data visualisation. This enables students to quickly interpret patterns in their daily activities. Graphical visualisations of the amount of time spent on various tasks can draw attention to periods of inefficiency and encourage users to review their schedules and take action. Users are more likely to consider their habits and take action to change them when information is visually intuitive. Another core principle is that it must maintain long-term user motivation over time; a successful PI system must have habit-forming mechanisms; otherwise, it fails to be effective. Streaks and progress awards are examples of features that can motivate users to engage with the system. According to psychology studies, users are more likely to maintain a habit when they receive some type of reward for regular participation [6]. Furthermore, including reminders at certain times can help support positive habits by reminding the user to interact with the system, which not only helps the user stay accountable but also helps gather more data to analyse their actions over time and provide more accurate reports on their habits. One of the main problems with PI systems is the accuracy of data input as well as the method of collecting data from the user. A system that requires time-consuming manual procedures to enter data can discourage users from using the system regularly. In order to avoid this, automated data tracking can be implemented using calendar applications and sensors. When manual input is necessary, the system should avoid making it unnecessarily complicated and time-consuming; for example, by introducing quick-entry options such as voice commands or predictive text makes it convenient for the user and encourages them to use the system on a regular basis.

\subsection{Introducing The PI system}
The Personal Informatics (PI) system we have developed offers personal insights into students' productivity patterns; our system is specifically made to help manage their time and reduce procrastination. The system achieves this by correlating productivity with screen time and gives the user graphical visualisations of how efficient they are with their time. The system measures productivity using 2 key metrics: task completion rate and time spent on tasks. Users will log tasks within the system and either toggle the tasks as either completed or not completed. The system , then, at the end of the day, will analyse the amount of tasks completed alongside the time taken on each task and analyse the trends in focused work versus screen time. Since our system relies solely on manual inputs, it prioritises user engagement and self-reflection as core elements of its effectiveness. To enhance the effectiveness of our PI system, we have incorporated several key principles, some of which were mentioned previously in the report. Data reflection within our system will allow for behaviour change. The system provides the users with a summary of completed versus incomplete tasks, allowing them to identify areas for improvement. Consistency and engagement are other principles which maintain user motivation. The system will have habit-forming techniques such as streak tracking and self-evaluation prompts, which will encourage users to reflect on their performance daily. Finally, the system emphasises the use of personalised insights and setting goals. Users reviewing their logged data can allow them to set personalised goals for improved time management. The system will enable them to recognise how poor time allocation and unproductive behaviours contribute to procrastination, guiding them toward more structured work habits.



\label{sec:intro}


\section{Agile Software Process Planning and Management (2 pages)}
Authors: Mandeep Thakur
\subsection{Introduction to Agile Approach}
To manage the development of our Personal Informatics (PI) system efficiently, we chose to use the agile methodology, enabling iterative refinements of our prototypes based on user feedback on requirements. This approach helped us better understand our target audience, allowing us to prioritise core features in our system. A major challenge we faced was the distribution of the workload and participation of the team. Despite having a team of ten members, only 6 individuals consistently engaged with the project. This imbalance required us to adjust our sprint planning strategically in order to have the project completed within the specified timeframe.

\subsection{Sprint Planning}
Due to certain circumstances, rather than evenly distributing the work among all active members of the team, we focused on the individual strengths of the six active members. We used several strategies to maintain productivity , including prioritising key tasks such as the User Interface design and database structures, which were the core features of our PI system. Another key approach was the strategic assignment of roles within the group. We asked people to choose tasks within the sprint that they were most confident in, allowing them to leverage their skills and reduce delays. The agile flexibility allowed us to dynamically adapt our approach , ensuring that the missing contributions did not affect the overall progress.
To manage our sprints several different platforms were used. Onenote was use for documenting tasks, sprint plans, and meeting notes. Github was used for version control and tracking code contribution, LaTeX was used for structured documentation and finally Discord for team communication. There were 2 sprints in total, with each sprint lasting two to three weeks, each with specific milestones.

\subsubsection{Sprint 1}
Our first sprint served as the basis for our project, focusing on initial research and core development. The planned tasks were documented and managed in OneNote. The main milestones of the sprint included researching the PI system, gathering user requirements through a Google Form, designing the UI layout, implementing task tracking, and setting up databases to store productivity and screen time logs. Each of these milestones contributed to the development of a functional prototype, allowing us to test the core features and collect user feedback. Throughout Sprint 1, daily progress was monitored through daily meetings on Discord, where members of the team provided updates and raised issues. Additionally, four formal scrum meetings (24/03, 28/03, 31/03) were held during the three-week sprint to review milestones and adjust sprint goals. To evaluate the sprint, we created another Google form to collect feedback from the students, which helped us refine our approach for sprint 2.
\subsubsection{Sprint 2}
The second sprint's purpose was to improve data analysis and provide meaningful insights of users' data through data visualisation.
\subsection{Importance Of Scrum Meetings}
Scrum meetings were essential throughout the development of our PI system, helping us prioritise features and keep the project on pace despite our small team size. During Sprint 1, meetings were used to finalise the core functionalities of our system and reallocate tasks when some team members became inactive. These meetings also helped us identify and resolve issues such as challenges with the Tkinter GUI implementation. In Sprint 2, the main focus was on refining data analysis and visualisation with also some adaptation to the UI based on user feedback collected from Sprint 1. These meetings were used..... In conclusion, these scrum meetings allowed us to prioritise features given our team size, improve communication and collaboration and adjust progress to meet deadlines effectively.
\label{sec:agile}


\section{Software Requirements Specification (5 pages)}
Authors:
\label{sec:requirements}
\subsection{Requirements Gathering}

\subsection{Use Cases}

\subsection{Functional Requirements}

\subsection{Non‑Functional Requirements}


\section{Design (5 pages)}
Authors:
\label{sec:design}
\subsection{UML Class Diagrams}

\subsection{Sequence Diagrams}

\subsection{Design Rationale}


\section{Software Testing (Verification) (2 pages)}
Authors:
\label{sec:testing}
\subsection{Test Plans}

\subsection{Test Results}


\section{Reflection and Conclusion (4 pages)}
Authors:
\label{sec:reflection}
\subsection{System Critique}

\subsection{Process Critique}


%=== REFERENCES (does not count towards page limit) ===
\newpage
\bibliographystyle{ieeetr}
\section*{References}
% \bibliography{references}

%=== APPENDICES ===
\newpage
\appendix

\section{Group Contribution Form}
% Insert completed GCF

\section{Meeting Minutes}
% Attach meeting minutes

\section{Interview Transcripts}
% Include transcripts

\end{document}
